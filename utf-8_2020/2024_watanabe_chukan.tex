\documentclass{jarticle}

\usepackage[ms]{pxchfon}% MSフォントを指定
\usepackage{twocolumn}

%\usepackage[dvi ps]{graphicx} %%画像を読み込む

\usepackage[dvipdfmx]{graphicx} %%画像を読み込む
\usepackage{here}%画像の配置場所の指定をプログラムの実行順に強制的に配置する
\newcommand{\setPicture}[1]{\includegraphics[width=0.9\linewidth]{/Users/watanabe_shouta/2024-chukan/picture/#1}}
  

\usepackage{subfigure}
\usepackage{amsmath}          %%genfrac http://www.biwako.shiga-u.ac.jp/sensei/kumazawa/tex/form006.html
%\usepackage{newtxtext,newtxmath}
\usepackage{ulem}             %%http://biwako.shiga-u.ac.jp/sensei/kumazawa/tex/ulem.html     uline,uuline,uwave,sout,xoutなど
\usepackage{multirow}
\usepackage{chukan2020}       %%最後に読み込むこと!(最後に読み込まないと\textwidthなどの設定が反映されない)

\pagestyle{empty} %ページ番号を入れるときにはコメントアウトする

\begin{document}

\linesparpage{50}

\title{
柔軟かつ屈曲可能な胴体を有する魚型ロボットの開発
}
\etitle{
 Development of Fish-Type Robot with Flexible and Bendable Torso
}
\author{
研究者 渡部 翔太\;\;\;
指導教員 中西 大輔
}
\eauthor{
Keywords: Fish-Type Robot, Flexible and Bendable Torso
}

\maketitle

\thispagestyle{empty}  %1ページ目にページ番号を入れるときにはコメントアウトする

%%%%%%%%%%%%%%%%%%%%%%%%%%%%%%%%%%%%%%%%%%%%%%%%%%%%%%%%%%%%%%%%%%%%%%%%%%%%%%%
\section{緒言}
水中・水上の推進システムにはスクリュープロペラを用いた推進方法や,魚を模したロボットによる尾びれ推進などがあげられる\cite{ichi}.
中でも尾びれ推進は周辺環境へ影響を与えず,加速性・旋回性に優れている.こういった点から尾びれ推進を用いた魚型ロボットの開発は生態
系調査や災害への支援といった面において注目されている\cite{ni}.
これに対して我々は,様々な魚型ロボットを研究・開発してきた.その中で昨年度の研究では,胴体部分を屈曲可能なリンク構造にすることで体を
しならせる魚らしい動きの実現にも成功している.また,リンク間にできる隙間によって水をうまくかけていないという問題を解決するために,
柔軟な外皮をかぶせたロボット(図\ref{fig:bandable-Torso})を開発した\cite{san}.しかし,外皮がリンクにうまく追従せず,遊泳性能
が落ちてしまった.そこで本研究では外皮をリンクにうまく追従させ,かつ魚らしい動きを実現できるロボットの開発を目指す.

%%%%%%%%%%%%%%%%%%%%%%%%%%%%%%%%%%%%%%%%%%%%%%%%%%%%%%%%%%%%%%%%%%%%%%%%%%%%%%%

\section{ワイヤ駆動式魚型ロボットについて}
昨年度卒業研究で開発されたワイヤ駆動式の魚型ロボットについて述べる.魚型ロボットの構造を図\ref{fig:structure}に示す.
ロボットは制御部,駆動部,胴体部の三つで構成され,頭部にあたる制御部にはマイコン,基板,バッテリが,えら部分にあたる駆動部にはプ
ーリ付きのサーボモータが,胴体部には弾性体,尾びれが配置される.
動作としてはサーボモータに取り付けられたプーリが左右に動き,ワイヤが巻き取られることによって,弾性体に取り付けられたリンクが引っ
張られ,弾性体をたわませる.それによって体をしならせ、水をかくことで遊泳する.
本研究ではこのロボットをベースとして新たなロボットを開発する.


\section{試作機の作成}
ロボットの開発にあたり,まずは昨年度開発されたワイヤ駆動式の魚ロボットを参考に試作機を作成することにした.
最初にロボットを構成する部品の作成を行った.頭部と胴体のリンク部分は3Dプリンタを用いて制作し,胴体部の弾
性体はレーザ加工機を使用して作成した.次に基板加工機を用いて回路製作を行った.また,機体の部品作製,回路
作成を行う過程で,使用するマイコンの動作確認とプログラム作成を行った.

マイコンの動作確認を行った後,試作
機の組み立てを行った.組み立て後の機体を図\ref{fig:fish-type}に示す.組み立て後,手に持って空中での動作
確認を行った.プログラム通り
%図の配置にHを使うとコードの順に無理やりねじ込もうとするので見出しと本文の間隔が引き延ばされる可能性アリ.注意すべし
\newpage

\begin{figure}[H]
   \centering
   \setPicture{fish-torso.jpg}
   \vspace*{-4mm}
   \caption{柔軟外皮を有するロボット\cite{ni}}
   \label{fig:bandable-Torso}
\end{figure}
\vspace{-4mm}
\begin{figure}[H]
   \centering
   \setPicture{tentativeschematic.png}
   \vspace*{-4mm}
   \caption{ロボットの構造(昨年度卒業研究)}
   \label{fig:structure}
\end{figure}
\vspace{-4mm}
\begin{figure}[H]
   \centering
   \setPicture{fish-copy.jpg}
   \vspace*{-4mm}
   \caption{開発した試作魚型ロボット}
   \label{fig:fish-type}
\end{figure}
\vspace{-4mm}

\noindent の正常な動作を確認することができた.%noindentを使えばHで図を配置した後に起きる強制的な改行を抑制することができる


その後実際に遊泳を行うために頭部の防水実験を行った.3回実施したが,2回は完全浸水し,1回は頭部下方のみ
浸水した.浸水シールの反応から考えると,おそらく頭部に着けているコネクタの隙間から浸水していると考えられる.
この問題に対処するために今後はコネクタの隙間をレジンで覆うなど防水対策を行い,そして電装部品を頭部に搭載し
て遊泳実験を行う.

%%%%%%%%%%%%%%%%%%%%%%%%%%%%%%%%%%%%%%%%%%%%%%%%%%%%%%%%%%%%%%%%%%%%%%%%%%%%%%%
\section{まとめと今後の予定}

本稿では昨年度開発されたロボットを参考に試作機を作成した.今後は防水対策を施した試作機を用いて遊泳実験を行
う.そして柔軟な外皮をロボットにかぶせ,完全防水可能かつ,遊泳性能を向上させた魚型ロボットの開発を目指す.
さらに最終的には柔軟な外皮によって生まれるしわを無くすために外皮表面に鱗を付け,さらなる遊泳性能の向上を目指す.


%%%%%%%%%%%%%%%%%%%%%%%%%%%%%%%%%%%%%%%%%%%%%%%%%%%%%%%%%%%%%%%%%%%%%%%%%%%%%%%

% \bibliography{mylib}
% \bibliographystyle{junsrt}


\begin{thebibliography}{99}

   \bibitem{ichi}
   平田宏一, 春海一佳, 瀧本忠教, 田村兼吉, 牧野雅彦, 児玉良明, 冨田宏. 魚ロボットに関する基礎的研究. 海上技術安全研究所報告, Vol. 2, No. 3, pp. 281-307, 2003.

   \bibitem{ni}
   高田洋吾, 中西志允, 荒木良介, 脇坂知行. Piv 測定と3 次元数値解析による小型魚ロボット周りの水の流動状態と推進能力の検討(機械力学, 計測, 自動制御). 日本機械学会論文集C 編, Vol. 76, No. 763, pp. 665–672, 2010.

   \bibitem{san}
   中西大輔, 石原康平, 柔軟外皮を有する飛び移り座屈駆動式魚型ロボットの開発. ロボティクス・メカトロニクス講演会2024, 2P2-B10, 2024.

\end{thebibliography}
\end{document}
