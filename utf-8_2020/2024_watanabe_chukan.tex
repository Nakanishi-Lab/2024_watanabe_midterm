\documentclass{jarticle}

\usepackage[ms]{pxchfon}% MSフォントを指定
\usepackage{twocolumn}

%\usepackage[dvi ps]{graphicx} %%画像を読み込む

\usepackage[dvipdfmx]{graphicx} %%画像を読み込む
\newcommand{\setPicture}[1]{\includegraphics[width=0.9\linewidth]{/Users/watanabe_shouta/2024-chukan/picture/#1}}
  

\usepackage{subfigure}
\usepackage{amsmath}          %%genfrac http://www.biwako.shiga-u.ac.jp/sensei/kumazawa/tex/form006.html
%\usepackage{newtxtext,newtxmath}
\usepackage{ulem}             %%http://biwako.shiga-u.ac.jp/sensei/kumazawa/tex/ulem.html     uline,uuline,uwave,sout,xoutなど
\usepackage{multirow}
\usepackage{chukan2020}       %%最後に読み込むこと!(最後に読み込まないと\textwidthなどの設定が反映されない)

\pagestyle{empty} %ページ番号を入れるときにはコメントアウトする

\begin{document}

\linesparpage{50}

\title{
柔軟な/屈曲可能な胴体を有する魚型ロボットの開発
}
\etitle{
 Development of Fish-Type Robot with Flexible/Bendable Torso
}
\author{
研究者 渡部 翔太\;\;\;
指導教員 中西 大輔
}
\eauthor{
Keywords: Fish-Type Robot, Flexible/Bendable Torso
}

\maketitle

\thispagestyle{empty}  %1ページ目にページ番号を入れるときにはコメントアウトする

%%%%%%%%%%%%%%%%%%%%%%%%%%%%%%%%%%%%%%%%%%%%%%%%%%%%%%%%%%%%%%%%%%%%%%%%%%%%%%%
\section{緒言}

水中・水上の推進システムにはスクリュープロペラを用いた推進方法や,魚を模したロボットによる尾びれ推進などがあげられる\cite{ichi}.スクリュープロペラ
は推進性能が高く,広く船舶などに用いられてきた.その一方で,スクリューは周辺環境への影響が大きく,水棲生物の生態系の調査や災害への支援に
用いられる推進システムとして適していない.それに対して,尾びれ推進は周辺環境へ影響を与えず,加速性・旋回性に優れている.こういった点から
尾びれ推進を用いた魚型ロボットの開発は生態系調査や災害への支援といった面において注目されている\cite{ni}.

これに対して先行研究では,様々な魚型ロボットを研究・開発してきた.その中で胴体部分を屈曲可能なリンク構造にすることで体をしならせる魚らし
い動きの実現にも成功している.昨年度の研究では,リンク間にできる隙間によって水をうまくかけていないという問題を解決するために,シリコンで
作成した柔軟な外皮をロボットにかぶせ,完全防水に成功した\cite{san}.しかし,外皮がリンクにうまく追従せず,遊泳性能が落ちてしまった.そこで本研究では
外皮をリンクにうまく追従させ,かつ魚らしい動きを実現できるロボットの開発を目指す.さらに体をしならせる時にできるしわも遊泳性能を低下させて
いると考えられるため,外皮に鱗を付けることでしわをなくし,さらなる遊泳性能の向上を目指す.

%%%%%%%%%%%%%%%%%%%%%%%%%%%%%%%%%%%%%%%%%%%%%%%%%%%%%%%%%%%%%%%%%%%%%%%%%%%%%%%
\vspace*{-1mm}
\section{デッドコピーの作成}
\subsection{コピー元の選定}
今回,研究を進めるにあたってまずは昨年度の研究で開発された魚ロボットのデッドコピーを作成することにした.デッドコピーとは既存の工業製品や商品などの構造・使用を
完全に,もしくはほとんどの部分で踏襲して複製した模造品のことである.
駆動方式に飛び移り座屈を用いると魚らしい動きを実現できない懸念があるため,今回は魚らしい動きの実現のためにワイヤ駆動型魚型ロボットをコピー元とした.(図\ref{fig:fish-type})

\subsection{ロボットの構造・動作}
コピー元の魚ロボットの構造を図\ref{fig:structure}に示す.ロボットは制御部,駆動部,胴体部の三つで構成され,頭部にあたる制御部にはマイコン,基板,バッテリが,えら部分にあたる駆動部にはプーリ付きの
サーボモータが,胴体部には弾性体,尾びれが配置される.頭部は正面から縦に分割できるようになっている.
動作としてはサーボモータに取り付けられたプーリが左右に動き,ワイヤが巻き取られることによって,弾性体に取り付けられたリンクが引っ張られ,弾性体をたわませる.それによって体をしならせ、水を書くことで遊泳する.


\begin{figure}[t]
   \centering
   \setPicture{fish.jpg}
   \vspace*{-4mm}
   \caption{コピー元のロボット}
   \label{fig:fish-type}
\end{figure}

\begin{figure}[t]
   \centering
   \setPicture{tentativeschematic.png}
   \vspace*{-4mm}
   \caption{ロボットの構造}
   \label{fig:structure}
\end{figure}

\subsection{デッドコピーの作成}
デッドコピーを作成するするために,昨年度のデータを用いて胴体部分の印刷,基板の印刷を行った.また,昨年度使用されたマイコンを用いた動作テストを空中で行い,正常に動作することを確認した.
しかし,防水実験を行った際に内部が浸水するという問題が起きた.これに対処するためにねじの締め付ける順番を対角線上に変更したり,頭部の分割面にやすり掛けを行い水平面にしたりして浸水対策を行う.

%%%%%%%%%%%%%%%%%%%%%%%%%%%%%%%%%%%%%%%%%%%%%%%%%%%%%%%%%%%%%%%%%%%%%%%%%%%%%%%
\vspace*{-2mm}
\section{まとめと今後の予定}
本稿では柔軟な,屈曲可能な胴体を有する魚型ロボットの開発のために昨年度開発されたロボットのデッドコピーを作成した.今後は防水対策を施したデッドコピーを用いて遊泳実験を行う.
そして柔軟な外皮をロボットにかぶせ,完全防水可能かつ,遊泳性能を向上させた魚型ロボットの開発を目指す.


%%%%%%%%%%%%%%%%%%%%%%%%%%%%%%%%%%%%%%%%%%%%%%%%%%%%%%%%%%%%%%%%%%%%%%%%%%%%%%%

% \bibliography{mylib}
% \bibliographystyle{junsrt}


\begin{thebibliography}{99}
   \bibitem{ichi}
   平田宏一, 春海一佳, 瀧本忠教, 田村兼吉, 牧野雅彦, 児玉良明, 冨田宏. 魚ロボットに関する基礎的研究. 海上技術安全研究所報告, Vol. 2, No. 3, pp. 281–307, 2003.

   \bibitem{ni}
   高田洋吾, 中西志允, 荒木良介, 脇坂知行. Piv 測定と3 次元数値解析による小型魚ロボット周りの水の流動状態と推進能力の検討(機械力学, 計測, 自動制御). 日本機械学会論文集C 編, Vol. 76, No. 763, pp. 665–672, 2010.

   \bibitem{san}
   中西大輔, 石原康平, 柔軟外皮を有する飛び移り座屈駆動式魚型ロボットの開発. ロボティクス・メカトロニクス講演会2024, 2P2-B10, 2024.

\end{thebibliography}
\end{document}